\documentclass[a4paper]{article}
\usepackage[margin=1in]{geometry}
\usepackage{amsmath}
\usepackage{amssymb}
\usepackage{lineno}
\usepackage{hyperref}
\usepackage{svg}
\usepackage{natbib}

\bibliographystyle{sysbio}

\newcommand{\CountFull}[1]{|#1|_\text{full}}
\newcommand{\CountEmpty}[1]{|#1|_\text{empty}}
\newcommand{\bigrig}{\texttt{bigrig}}
\newcommand{\decj}{DEC[+J]}
\newcommand{\rand}[2]{#1 \land #2}
\newcommand{\ror}[2]{#1 \lor #2}
\newcommand{\rneg}[1]{\neg #1}
\newcommand{\rxor}[2]{#1 \oplus #2}
\newcommand{\rLshift}[2]{#1 \ll #2}
\newcommand{\rRshift}[2]{#1 \gg #2}

\title{Supplemental material for \bigrig{}}
\author{Ben Bettisworth and Alexis Stamatakis}

\linenumbers

\begin{document}
\maketitle

\section{DEC-ishSSE}

We want to implement in \bigrig{} an SSE style model, as these models are becoming more popular for inference of
ancestral ranges and other states.
The obvious model implement would be GeoSSE \citep{goldberg2011}, which is an SSE model specifically developed for
historical biogeography.
However, the number of parameters for the GeoSSE model scales exponentially with the number of regions.
This makes it impractical, both to implement in software for technical reasons, and to use.

An alternative model which has significantly fewer parameters is FIG \citep{landis2022}.
However, while the number of parameters is indeed much smaller, the relationships \textit{between} those parameters is
much more complicated.
As such, the model is challenging to implement, and to ensure that the implementation is correct, add the relationships
between the parameters obfuscates the operation of the underlying mechanics.
Therefore, we will instead seek to develop our own model which meets our criteria, which are:
\begin{enumerate}
  \itemsep 0pt
  \parskip 0pt
  \item Model the cladogenesis process as tree growth;
  \item Incorporate state specific rates;
  \item And is simple to implement.
\end{enumerate}
That is, the model needs to be able to sample a tree along with internal states, and the specific probability of
cladogenesis, extinction, or dispersal is state specific.
Finally, the model should be relatively simple to implement as software while also containing all the fundamental pieces
of the GeoSSE model.
To this end, we present DEC-ishSSE, a toy model which should be used as a proof of concept for implementing software.
Here, a toy model means that it is not intended for use with ``real'' data, but instead should be used for validating
software and inference methods, as a stepping stone to more complicated and realistic models.

The main thrust of DEC-ishSSE is that each region has 6 parameters specified, one for each rate of a normal
\decj{} model. 
Then, states derive their specific rates from the rates of its corresponding region.
Each region rate is either an ``in'' or a ``to'' rate, where an ``in`` rate is the rate of of the event occurring
\textit{in} the region, and an ``to'' rate is the rate of the event occurring \textit{to} the region from all other
regions.
In total, DEC-ishSSE has 6 parameters for each region, which are listed in Table~\ref{tab:decish-rates}.

\begin{table}
  \begin{center}
    \begin{tabular}[c]{l|l|l}
      \hline
      \multicolumn{1}{c|}{\textbf{Parameter}} & 
      \multicolumn{1}{c|}{\textbf{Description}} & 
      \multicolumn{1}{c}{\textbf{Type}} \\
      \hline
      $a$ & Rate of Allopatric Cladogenesis & In \\
      $s$ & Rate of Sympatric Cladogenesis & In \\
      $y$ & Rate of Sympatric (Singleton) Cladogenesis & In \\
      $j$ & Rate of Jump Events & To \\
      $d$ & Rate of Dispersion Events & To \\
      $e$ & Rate of Extinction Events & In \\
      \hline
    \end{tabular}
  \caption{Per region parameters for the DEC-ishSSE model.}\label{tab:decish-rates}
  \end{center}
\end{table}

The rate of an ``in'' event $I$ for a specific range (state) $R$ is given as
$$
I(R) = \sum_{r \in R}
\begin{cases}
  I(r)  &\text{ if } \CountFull{r} = 1\\
  0     &\text{ otherwise.}
\end{cases}
$$
Here $I(r)$ is simply the rate of event type $I$ for region $r$.
Similarly, the rate of an ``to'' event  $T$ is 
$$
T(R) = \CountFull{R} \times \sum_{r \in R}
\begin{cases}
  T(r)  &\text{ if } \CountFull{r} = 0\\
  0     &\text{ otherwise.}
\end{cases}
$$
Likewise, $T(r)$ is the rate of event type $I$ for region $r$.
As mentioned above ``to'' events represent when range evolution happens from region to region.
To model this, we include the factor $\CountFull{R}$ to the rate of out events, as each full range is capable of acting
as a source region to the destination region.

A pair of cases are required when computing $I$ based on if $R$ is a singleton or not, i.e. $\CountFull{R} = 1$.
In the non-singleton case, the copy cladogenesis is disallowed, i.e. the rate of copy events is $0$.
In the singleton case, sympatric and allopatric cladogenesis is disallowed, and copy cladogenesis is allowed, i.e.
$s=a=0$.

Once the rate for a range has been computed, the evolution of that range is governed by an exponential process with
rate equal to the sum of all event rates.
Once an event occurs, the event type and range is drawn according to the proportional weight of the range and event.

\subsection{Using DEC-ishSSE in practice}

If one insisted on using DEC-ishSSE with real data, we would recommend the following two restrictions:
\begin{itemize}
  \itemsep 0pt
  \parskip 0pt
  \item $y=s$ for all regions;
  \item and $j=0$ for all regions.
\end{itemize}
In particular, jump events allow for regions that are not part of the parent range to be present in one of the daughter
ranges.
This confounds the processes of dispersion and allopatric cladogenesis, and prevents DEC-ishSSE from being a sub-model of
GeoSSE.
We suggest the constraint the constraint $s=y$ as it makes the model more consistent with the principle that each region
has some rate of an event occurring.
Given that sympatric cladogenesis and copy cladogenesis are functionally the same event, having the two events
represented by different rates is inconsistent.
Additionally, this constraint improves the biological plausibility of the model, as it seems unlikely that a population
is more or less likely to undergo sympatric cladogenesis based on whether or not the population is endemic to a single
region.

With these restrictions, DEC-ishSSE becomes a strict sub-model of both GeoSSE and FIG.

\bibliography{references}

\end{document}
